\documentclass{report}
\usepackage{hyperref}
\usepackage{listings}
\usepackage{amssymb}
\lstset{
  basicstyle=\ttfamily,
  mathescape
}
\hyphenation{Gram-Schmidt}

\newcommand{\Input}{\mathbf{input}\ }
\newcommand{\For}{\mathbf{for}\ }
\newcommand{\End}{\mathbf{end}\ }

\pagestyle{empty}
\begin{document}


% Unofficial translation excerpt based on
% https://cs.wikipedia.org/wiki/Gramova-Schmidtova_ortogonalizace

\noindent\textbf{Gram-Schmidt process}

\noindent{\small\rule{0pt}{2.7ex}Based on
  \texttt{https://cs.wikipedia.org/wiki/Gramova-Schmidtova\_ortogonalizace}}

%\vspace{1ex}\noindent{Algorithm}

\begin{lstlisting}
00: $\Input a_1,\ldots,a_n$
01: $r_{1,1} := \|a_1\|_2$
02: $q_1 := a_1/r_{1,1}$
03: $\For k := 2,\ldots,n$
04:   $p := a_k$
05:   $\For j := 1,\ldots,k-1$
06:      $r_{j,k} := q_j^Tp = \langle p,q_j\rangle$
07:   $\End$
08:   $\For j := 1,\ldots,k-1$
09:      $p := p - q_jr_{j,k}$
10:   $\End$
11:   $r_{k,k} := \|p\|_2$
12:   $q_k := p/r_{k,k}$
13: $\End$
\end{lstlisting}

\noindent The Gram-Schmidt process as described above is referred to
as \textbf{classical Gram-Schmidt (CGS)} and is newer than the
original variant today referred to as \textbf{modified Gram-Schmidt
  (MGS)}. MGS is obtained from CGS simply by removing lines 07 and 08,
that is by joining both inner loops. Iterative orthogonalization can
be used to avoid small deviations from orthogonality in the
Gram–Schmidt process. The variant repeating both inner loops twice is
known as \textbf{iterated classical Gram-Schmidt (ICGS)}.

~

% Octave:   A = [ones(1,20); eye(20)*1e-7]; cond(A)
%
\noindent Let's consider \textbf{L\"auchli matrix} defined as
%
$$
A = \left[\begin{array}{ccc}
    1    & \ldots & 0 \\
    \rho &        & 0 \\
    & \ddots      &   \\
    0    &        & \rho \\
  \end{array}\right] \in \mathbb{R}^{(n+1)\times n},\quad n=20,
  \quad\rho=10^{-7},\quad \kappa_2 \approx 4.47\times 10^{-7},
$$
%
where $\kappa_2$ is condition number of matrix $A$. Considering
standard hardware arithmetics for double precision $\epsilon \approx
2.22\times 10^{-16} $, then loss of orthogonality corresponding to the
above given algorithms applied to L\"auchli matrix is given in the
second column in the following table. In the third column is given
general formula valid for arbitrary matrix $A.$

~

\begin{center}
\begin{tabular}{|l|cc|}\hline
            & \multicolumn{2}{c|}{Loss of Orthogonality} \\
  Algorithm &  L\"auchli Matrix & General Matrix \\
  \hline
  \hline
  CGS  \rule{0pt}{2.4ex} % add some vertical space needed in the third column
       & $2.2\times 10^{-2} $ & $\kappa_2^2(A)\epsilon_M$ \\
  MGS  & $2.2\times 10^{-9} $ & $\kappa_2  (A)\epsilon_M$ \\
  ICGS & $2.4\times 10^{-16}$ & $\epsilon_M$ \\
  \hline
\end{tabular}
\end{center}

\end{document}
